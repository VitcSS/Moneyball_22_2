\section{Introdução}
No contexto dos esportes coletivos é bastante complexo definir  
a melhor formatação para um time de elite, tendo em vista  
que parâmetros fora do ambiente de jogo tem um peso imprevisível nos resultados  
observados em campo. Tentando tornar esses eventos caóticos, previsíveis estrategistas e estatísticos  
especializados em esportes desenvolveram métricas quantitativas e qualitativas que descrevem o esporte  
de uma forma que seja empírica e fiel. Estas  

Mesmo com anos de estudo, o gap entre os parâmetros medidos e um modelo suficientemente assertivo de predição 
de vitórias esportivas ainda é abissal. Uma forma que vários estudiosos encontraram de tornar o processo de mais  
simples é o uso das métricas individuais dos jogadores de forma a ranquear os times pelas suas qualidades e defeitos. 

Esse tipo de estudo estatístico do esporte foi primeiramente feito para o baseball em 1858 quando o jornalista esportivo Henry Chadwick o box score, 
que disponibilizava as estatísticas individuais de cada jogador, assim como do time após cada partida. O estudo se intensificou no século 20  
com a fundação da Society for American Baseball Research que desenvolveu o que ficou conhecido como Sabermetrics, o que é definida como uma metodologia  
para a compreensão do baseball através de fatos objetivos e mensuráveis. 

Tem se tornado muito comum o uso desse tipo de estatística é para formação ou preenchimento de vagas abertas em times esportivos.  
Isso por si só é suficiente para modelar um problema de otimização mono-objetivo, entretanto, nas grandes ligas profissionais  
não somente o resultado é importante, mas também o quanto esse resultado custou a organização na temporada. Isso torna a questão da formação de um time  
bem mais complexo, pois trata-se não só do melhor time, mais sim do melhor time que se pode comprar, gastando o mínimo possível.  

O primeiro exemplo de sucesso da Sabermetrics aplicada a formulação de times, apelidada na época de Moneyball, foi feita pelos Oakland Athletics em 2002, aonde  
com um gasto de mais de um milhão de dólares menor por jogo ganho, alcançaram o mesmo número de vitórias do New York Yankees, 103. Esse fato chamou a atenção de diversas organizações dentro e fora do baseball. 
Sendo alguns exemplos o futebol, o vôlei e o basquete, que será o foco do modelo descrito neste trabalho. 

