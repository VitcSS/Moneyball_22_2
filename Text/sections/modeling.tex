\section{Modelagem}

O objetivo é utilizando a base de dados de atletas da temporada regular da NBA 2021-22 e o mapa salarial dos 
atletas para temporada 2022-23, formular o melhor time possível, gastando o mínimo possível. O time em questão 
será formado por doze jogadores, dirtribuídos na entre as posições da seguinte forma.
\begin{table}[h]
    \centering
    \begin{tabular}{|c|c|c|}
    \hline
    \textbf{Posição} & \textbf{Sigla} & \textbf{Total de Jogadores} \\ \hline
    Armador          & PG             & 3                           \\ \hline
    Ala-armador      & SG             & 2                           \\ \hline
    Ala              & SF             & 2                           \\ \hline
    Ala de força     & PF             & 3                           \\ \hline
    Pivô             & C              & 2                           \\ \hline
    \end{tabular}
\end{table}
\subsection{Função Objetivo}
O problema possui duas funções objetivo uma que define a qualidade do time que foi montado e a outra que define o custo total 
do time. A função de avaliação pode ser descrita da seguinte forma:
\begin{equation}
    f_{1}(x) = \frac{1}{N}\sum_{i=1}^{N}\sum_{j=1}^{M}P_{j}E_{j}(i)x_i
\end{equation}
Aonde :
\begin{itemize}
    \item \(i\) é o índice númerico que representa o jogador
    \item \(x_i\) é um vetor binário que indica se o jogador pertence ou não ao time formulado
    \item \(E_1(i)\) é o total de pontos por jogo
    \item \(E_2(i)\) é o número de assitências por jogo
    \item \(E_3(i)\) é o número de roubos por jogo 
    \item \(E_4(i)\) é o número de bloqueios por jogo 
    \item \(E_5(i)\) é o número de rebotes por jogo
    \item \(E_7(i)\) é o número de turnovers por jogo
    \item \(E_8(i)\) é o número de faltas por jogo
    \item \(P_{j}\) é um vetor de contantes de peso para cada um dos parâmetros contabilizados, aonde :
    \begin{itemize}
        \item \(P_{j} > 0, j \leq 6 \)
        \item \(P_{j} < 0, 7 \leq j \leq 8\)
    \end{itemize}
    \item \(M\) é o total de parâmetros utilizados
    \item \(N\) é o total de jogadores
\end{itemize}

Todos os valores serão usados pós normalização para evitar que valores muito grandes como a pontução, quando comparado ao número de faltas, por exemplo,
force uma tendência sobre o modelo.

O custo do time poderá ser descrito da seguinte forma :

\begin{equation}
    f_2(x) = \sum_{i=1}^{N}S(i)x_i
\end{equation}
Aonde :
\begin{itemize}
    \item \(S(i)\) é uma função que para um índice de indetificação do jogador, retorna o salário pago a ele
    \item \(x_i\) é um vetor binário que indica se o jogador pertence ou não ao time formulado
\end{itemize}
\subsection{Restrições}
As únicas restrições usadas nesse modelo serão as de quantidade de jogadores por posição já expressadas anteriormente.
\subsection{Modelagem Formal}
\begin{equation}
    \begin{cases} 
        \max f_{1}(x) = \frac{1}{N}\sum_{i=1}^{N}\sum_{j=1}^{M}P_{j}E_{j}(i)x_i \\
        \min f_2(x) = \sum_{i=1}^{N}S(i)x_i
    \end{cases}
\end{equation}
\begin{equation*}
    \text{Sujeito a :} \\
    \begin{cases}
        \sum^{i=1}_{N}P_g(i)x_i \leq 3 \\
        \sum^{i=1}_{N}S_g(i)x_i \leq 2 \\
        \sum^{i=1}_{N}S_f(i)x_i \leq 2 \\
        \sum^{i=1}_{N}P_f(i)x_i \leq 3 \\
        \sum^{i=1}_{N}C(i)x_i \leq 2 
    \end{cases}
\end{equation*}
Por se tratar de um sistema multivariável e não completamente mapeado é correto assumir que tratasse de um problema de otimização linear multimodal, pois é correto assumir
que existam diversas combinações possíveis de times. O algoritmo que se pretende usar seria um algoritmo evolucionário modelado especificamente para 
o problema em razão do número de variáveis que o sitema apresenta.

